\batchmode
\documentstyle[11pt]{article}
\setlength{\textheight}{9.5in}
\setlength{\textwidth}{6.5in}
\setlength{\topmargin}{-0.5in}
\setlength{\oddsidemargin}{0.0in}
\setlength{\evensidemargin}{0.0in}
\setlength{\parindent}{0.in}
%\setlength{\parskip}{0.20in}
%\setlength{\itemsep}{0.10in}
%\setlength{\parsep}{0.20in}
%\setlength{\topsep}{0.10in}
 
%\setlength{\textwidth}{6.8in}
%\setlength{\textheight}{10.2in}
%\setlength{\topmargin}{+0.25in}
\begin{document}
%\begin{center}
%\end{center}
\vspace{1.5in}
\begin{center}
%{\Huge\bf {\sc MED}103: a laser-plasma\\
{\Huge{\bf A user guide \\for the laser-plasma simulation code: MED103}}
\end{center}
\begin{center}
{\LARGE A Djaoui}
\end{center}
\begin{center}
{\Large Central Laser Facility, Rutherford Appleton Laboratory}
\end{center}
\vspace{1.5in}
\begin{center}
ABSTRACT
\end{center}


Many features have been added  to the
laser plasma simulation code {\bf Medusa} \cite{cpc} over the last few years. 
This was renamed {\bf Med101} \cite{med101} in 1989 and now {\bf Med103}.
The new features in the latest version include: 
the coupling of the energies involved  in the atomic processes of excitation, ionization, etc... 
(in the non-LTE time-dependent average atom model)  to the
free electron energy balance equation, which allows realistic
simulations of rapidly heated plasmas found in  short-pulse high-intensity
laser-produced plasmas. 
The new version also  includes
a high-field corrections to inverse bremsstrahlung absorption,
tunnel ionization, a non-local  electron heat conduction subroutine
as well as a package for the modelling of collisional  in addition to recombination x-ray lasers.
Complete details for running this
code are given including: an explanation of the
input parameters, instructions for running on the Rutherford
Appleton Laboratory DEC ALPHA (AXPRL2), UNIX farm (OSFSERV) and the 
superscalar computer (COLUMBUS).
The code is capable of simulating a wide range
of laser-produced plasma experiments in 1 D planar, cylindrical or spherical
geometry with a choice of ionization and equation of state models.
  


\newpage 
\tableofcontents
\newpage
\section{Introduction}
This document outlines recent modifications to the laser
fusion code MED101 \cite{cpc,med101} (now renamed MED103).
Like MED101, MED103 is a one dimensional Lagrangian hydrodynamic
code which is capable of modelling a wider range of experiments.
Improvements to the physics in the program include 
\newline
1) Coupling of the energies involved in excitation and ionization,
when using the time-dependent average atom model,
to the free electron energy balance equation\cite{djaoui1}. 
This allows realistic simulations  of, for example, 
high-Z targets where a large fraction 
of the absorbed laser energy is consumed in the processes of
excitation and ionization of the plasma. 
\newline
2) Calculation of Ne-like collisional X-ray laser gain \cite{djaoui1} in 
addition to 
H-like, Li-like and Na-like recombination schemes.
Atomic data \cite{Bhatia}
is available in the code for 27 levels 
of 4 Ne-like ion configurations
($2s^2 2p^6, 2s^2 2p^5 3s, 2s^2 2p^5 3p, 2s^2 2p^5 3d$)
and for the following elements:
Si(Z=14), 
Ar(Z=18), Ti(Z=22), Fe(Z=26), Ge(Z=32) and Kr(Z=36).
\newline
3) A high field correction to inverse bremsstrahlung, 
a non-local heat conduction subroutine \cite{djaoui2} as well
as the option to switch on tunnel ionization (This is useful 
for the simulation of short pulse high intensity laser interaction with gas 
targets).
\newline
4) More control over the initial conditions of the beam/target and over the simulation.
For example it is now possible to specify the maximum time step DTMAX used by the
code (useful for the simulation of very short pulses). 
It is also possible to restart the code from where it left off in a previous run 
with the possibility of changing irradiation conditions (laser wavelength for example).
\newline
5) When dealing with
multiple Gaussian pulse irradiation  with largely different pulse lengths ( ie.  
a femtosecond pulse following  a nanosecond ASE prepulse), 
it is no longer necessary to use a small value of DTMAX
to perform the simulation. 
The code 'looks ahead' for short pulses and adjusts the time step so as to sample
the short pulse adequately.

 
The source code for MED103 has also been 'cleaned-up' and some 
errors corrected.
MED103 now compiles and runs 
on COLUMBUS, the UNIX farm (OSFSERV), and DEC 
ALPHA (AXPRL2) without any modifications.
 
There are four different NAG/GKS graphics programs to post-process MED103
output:
\newline
1) FLIP3 --- this is the most useful and it plots 
the hydrodynamic variables (velocity, mass density, electron
density, pressure, electron and ion temperatures,
and average ionization) vs.\ distance
for different times, and also the hydrodynamic variables
(cell coordinates plus velocity etc.)\ vs.\ time
for each cell
\newline
2) ION3 --- this plots the ground state number density
of the different ionization stages vs.\ distance at
different times during the interaction.
ION3 also displays the zoning of the run. This can be useful to
ensure that there are enough cells in the regions of interest
\newline
3) XRLREC3 --- this plots the recombination X-ray laser gain vs.\
distance at different
times, and also the space-integrated gain for different
lines (i.\ e.\ alpha, beta etc.)\ vs.\ time as would
as measured by experiment.
\newline
4)XRLCOL3
this plots the Ne-like collisional X-ray laser gain vs.\
distance at different
times.
\newline
 
Each graph has a panel at the top containing details
of the input data used for that particular run.
The graphics packages were written initially to run on the old IBM and use
NAG GKS routines. They produce postscript files  which can 
be viewed on any X-windows display using {\bf ghostview} and either
printed at Rutherford or transferred to the user's local site and
printed on a local postscript printer. 

\newpage 
\section{Getting started}

If you don't already have an account on the Unix farm (OSFSERV) or the DEC ALPHA
(AXPRL2) contact the Theory
and Computation group at the Central Laser Facility (telephone:
01235 445481/446344). Information and documentation on the available computing
services can be found on the word wide web (http://www.cis.rl.ac.uk/services.html)

In the following sections 
commands that exist in batch or script files or commands that can be typed in
by the user will be printed
\begin{verbatim}
IN THIS TYPEFACE or this typeface
\end{verbatim}

\subsection{ A note  to users of MED101}
When implementing any modification to the code, I have tried to do it
in a way which leaves the code upward compatible (i.e it can
run old data sets without any changes). This philosophy has been
adhered to, except for the use of the variable SAHA. In MED101,
when the X-ray laser calculation is switched on (ICXRL=1) the code
automatically uses the time dependent non-LTE ionization (Z*) from the solution
to the non-LTE average atom model rate equations. {\bf In MED103 it is
necessary to set (SAHA=2.0) for the code to calculate the ionization
from the non-LTE average atom rate equations independently of  the value of ICXRL}.
The ionization energy is  also calculated and included in the solution
to the free electron energy balance equation.
With SAHA=2,   either the perfect
gas (NLPFE=T) or the degenerate gas (NLPFE=F, STATE=0) equations of state
may be used.

If the user mixes the non-LTE model (SAHA=2) with the LTE
Thomas Fermi equation of state
(NLPFE=F and STATE=1, 2 or 3) either inadvertently or purposefully
the code will calculate the ionization
energy from the non-LTE model while the pressure is calculated from the TF model.
This choice is arbitrary and is {\bf thermodynamically inconsistent}. Therefore this is not advised.
Typical settings of SAHA, NLPFE, STATE, ICXRL are given in the examples
in this manual.
                                                            
An example input data file  (FPAP2.DAT) for the modelling of the
x-ray laser experiment is shown next. A similar example  was also given in the med101
manual\cite{med101}.
\begin{verbatim}
FLUORINE FIBRE (med103 version: saha=2.0)
0.53 microns, 3.5e12 W/m/rad, 70 ps FWHM,
8 micron diameter, perfect gas EOS
************************************************
 $NEWRUN
 XAMDA1=0.53E-6, GAUSS=1.0,    ANPULS=1.0,  TOFF=1.0,
 PLENTH=4.2E-11, PMAX=3.50E12, PMULT=2.381,
 NGEOM=2,        PIQ(55)=0.0,  TEINI=300.0, TIINI=300.0,
   MESH=50,   RINI=4.E-06,    RHOGAS=2635.0, RF1=0.9000,
   XZ=9.0,    XMASS=19.0,     FNE=1.0,
     ZGLAS=0.0, DRGLAS=0.0E-06, ROGLAS=0.0,    RF2=0.9999,
     XZ2=0.0,   XMASS2=0.0,     FNE2=1.0,
      ZPLAS=0.0, DRPLAS=0.00E-0, ROPLAS=0.0,    RF3=0.9999,
      XZ1=0.0,   XMASS1=0.0,     HYDROG=0.0,
 NPRNT=-100, TSTOP=1.E-09,  NRUN=999999,
 NP3=1,     NLEMP=F,
 NLBRMS=T,  FLIMIT=10.0,
 ANABS=0.2, FHOT=0.0,       FTHOT=-1.0,    RHOT=1.0,
 STATE=0.0, NLPFE=t,       saha=2.0,
 ICXRL=1,  IFRSTA=1, ILOSTA=0,   IHISTA=3,
 ISTAGE=1, IDFLG=0,  ROPMUL=1.0,  ITBFLG=1, ISTFLG=0,
 IPUFLG=0, TPON=2.15E-10,        TPOFF=2.85E-10,
 NLP=1,    NUP=3,    RMPD=0.002, DLAMDA=81.0E-08,
 NLMAX=2,  FLSHORT=0.0001,       FLLONG=0.55,
  nlres=f,
 $END
\end{verbatim}


\subsection{What is needed to run the code}
 
To run MED103 you will need an input data file. An example for the simulation
of a recombination x-ray laser experiment was given above. Another example
(AL100FS.DAT) is given below. 
This specifies an Al target which is irradiated by a 100 fs, $10^{17} W/cm^2$ pulse which is
preceeded by an ASE prepulse of 1 ns duration and $10^{11} W/cm^2$ peak intensity. Both pulses peak at 
1 ns (simulation time). The meaning of all
the input parameters is explained in 'Explanation of the input parameters' section. A third example 
for the simulation of a Ne-like x-ray laser experiment is also given in 'Changing the input data' section .

\begin{verbatim}
AL100FS: Al TARGET  PLANAR GEOM 0.268UM
2 PULSES: 100 FS, 1.E17 W/CM^2 ON TOP OF 1NS 1.E11 W/CM2 ASE
DEGENERATE  GAS, E BALANCE, ANABS=0.25
************************************************************************
 $NEWRUN
 XAMDA1=0.268E-6, TON=0.,  TOFF=1.,
 GAUSS=1.0, ANPULS=2.0,
 PMAX=1.E15, PLENTH=0.6E-09, PMULT=1.6666666667,
 XASER1(38)=1.E21, XASER1(42)=60E-15, XASER1(46)=16666.66667,
  NGEOM=1,        PIQ(55)=2.0,
   MESH= 65,   RINI=4.5E-6,    RHOGAS=2700., RF1=0.9999,
   XZ=13.,    XMASS=27.,   FNE=1.0,
    ZGLAS= 20., DRGLAS=.5000E-06, ROGLAS=2700.,  RF2=0.8711,
    XZ2=13.0,   XMASS2=27.00,   FNE2=1.0,
     ZPLAS=00., DRPLAS=0.00E-06, ROPLAS=2700.,  RF3=0.9999,
     XZ1=13.,   XMASS1=27.,   HYDROG=0.0,
   TEINI=11604.5000,  PIQ(86)=11604.50,  PIQ(89)=11604.50,
   TIINI=11604.500,  PIQ(87)=11604.50,  PIQ(90)=11604.50,
   PIQ(92)=1.,    PIQ(93)=1.,     PIQ(94)=1.,
   PIQ(85)=1.,    PIQ(88)=1.,     PIQ(91)=1.,
 NLHF=T, NLTNL=T, NLOMT3(10)=T, FLIMIT=10.,
 NLBRMS=T, 
 NLCRI1=T,  ANABS=0.25,   FHOT=0.0, FTHOT=-1.0, RHOT=1.0,
 NLPFE=F,    STATE=0.0, SAHA=2.0,
   ICXRL=0,  ISTAGE=1,
   IFRSTA=1, ILOSTA=0,   IHISTA=3,
   IDRFLG=0, DRMUL=1.0,
 NLRES=F,
 NPRNT=100.,  TSTOP=1.1E-9, NRUN=999999, DTMAX=1.E-11,
 NP3=1,     NLEMP=F,
 $END
\end{verbatim}
 
You will also need a  command  (DCL) file to run on the DEC ALPHA 
or  a script file to run on the 
UNIX farm or COLUMBUS. Examples of these files are given in the following  sections.
Copies of all the files  listed in this manual can be obtained from directory DISK$\_$USER3:[AD1.PUBLIC]
on machine AXPRL2.RL.AC.UK.

\subsection{Running on the DEC ALPHA}
The TCP/IP address for DEC ALPHA is AXPRL2.RL.AC.UK (or 130.246.12.10). In addition to the input
data file,
a DCL command file is needed to run med103
on  the DEC ALPHA. An example   is listed below. 

\begin{verbatim}
! COMMAND FILE MED103.COM TO RUN MED103 ON AXPRL2
$SET DEFAULT DISK_USER3:[AD1.MED103]
$FORTRAN/REAL_SIZE=64/ALIGN=ALL MED103
$LINK/EXECUTABLE=MED103 MED103
$ASSIGN AL100FS.DAT                             FOR005
$CREATE/DIRECTORY WEEK_DISK:[PUBLIC.WEEK.AD1]
$ASSIGN WEEK_DISK:[PUBLIC.WEEK.AD1]OAL100FS.OUT FOR006
$ASSIGN WEEK_DISK:[PUBLIC.WEEK.AD1]XAL100FS.OUT FOR011
$ASSIGN WEEK_DISK:[PUBLIC.WEEK.AD1]IAL100FS.OUT FOR012
$ASSIGN WEEK_DISK:[PUBLIC.WEEK.AD1]FAL100FS.OUT FOR013
$ASSIGN WEEK_DISK:[PUBLIC.WEEK.AD1]RAL100FS.OUT FOR015
$DELETE *.LIS;*
$DELETE *.OBJ;*
$DELETE *.MAP;*
$RUN MED103
$DELETE MED103.exe;*
$DEASSIGN FOR005
$DEASSIGN FOR006
$DEASSIGN FOR011
$DEASSIGN FOR012
$DEASSIGN FOR013
$DEASSIGN FOR015
$EXIT
\end{verbatim}

The first line is a comment line. The next line 
sets the default working directory to the directory where
MED103.FOR and the input data are kept on the DEC ALPHA. You will have to change 
this to point to the directory where you keep your files.
The next two lines compile and link 
the code. the executable file is given the name  MED103.
The next line assigns an input data file  to fortran stream 5
so it can be read by the code.
MED103 reads its input from stream 5. In
this case the input file is AL100FS.DAT.

The next line creates a subdirectory on WEEK$\_$DISK for writing the output.
In this version of the command file all the output from the code is 
written in the temporary directory WEEK$\_$DISK. Files in this directory are inspected at
midnight each night and are deleted if they are more than 7 days old.
If you want to keep any of the files 
in the WEEK directory you should copy them to your own disk space.
The next 6 lines assign the streams to various
output file names. 
MED103 produces a number of output
files depending on the input. Data is always output
on stream 6 (OAL100FS.OUT in this case), stream 13
(FAL100FS.OUT) and stream 15 (RAL100FS.OUT). 
OAL100FS.OUT will contain
details of the input data and defaults
set by the program plus all the output data in an easy-to-read form
(see section 3 for some sample output).
FAL100FS.OUT contains the hydrodynamic
variables in a form suitable for the graphics package FLIP3.
Stream 12 (IAL100FS.OUT)
contains the ionization data in a form suitable for ION3.
Stream 11 (XAL100FS.OUT) contains the X-ray laser output in a
form suitable for the graphic package XRLREC3 or XRLCOL3.
Streams 15 is used for the restart feature.
The code dumps all variables specifying the target on stream 15 (RAL100FS.OUT) (every time
it produces an output on stream 6). 
When restarting a previous run the code reads the necessary variables from stream 15. 

The next 3 lines clear some disk space by deleting
files which were created during the compilation and link stage and
are no longer needed. The next line runs the programme.
The following line deletes the executable file MED103.
If you do not want to recompile the code every time you run it,
you can keep the executable file MED103 by putting a comment (!)
in front of it. The next 7
lines deassign all the fortran streams and finally  the jobs terminates.

Batch jobs are submited to the DEC ALPHA using the standard DCL
command SUBMIT. 
To run this DCL command file as a batch job you will have to enter something like
\begin{verbatim}
SUBMIT/QUEUE=LONGQ/NOTIFY MED103
\end{verbatim}

Almost immediately you will be sent a message.
\begin{verbatim}
JOB MED103 (QUEUE LONGQ, ENTRY 205) PENDING
     PENDING STATUS CAUSED BY QUEUE STOPPED STATE
\end{verbatim}

205 is the job number. If for some reason you decide to cancel
this job type
\begin{verbatim}
DELETE/ENTRY=205 LONGQ
\end{verbatim}
If the job is already executing you should specify LONGQ$\_$EXEC instead of LONGQ as in
\begin{verbatim}
DELETE/ENTRY=205 LONGQ_EXEC
\end{verbatim}

When the job starts running a LOG file
MED103.LOG is created in your top directory.
Any error messages (total amount of CPU time used when the job finishes)
are written to this file.

You can find out about the status of your batch job by typing
\begin{verbatim}
SHOW QUEUE/BATCH
\end{verbatim}

The queue LONGQ has a maximum time limit of 6 hours.
To allow for batch jobs requiring different resources,
a number of different queues are available.
You can  find out the names, limits etc ...  of the other queues by
typing 
\begin{verbatim}
SHOW QUEUE/BATCH/FULL
\end{verbatim}
this will list all the queues and also give
you the maximum amount of CPU time allowed on any one queue.

When the job has finished you will be sent a message such as
\begin{verbatim}
JOB MED103 (QUEUE LONGQ_EXEC, ENTRY 250) COMPLETED
\end{verbatim}
Any output files will be sent to the files specified in the ASSIGN statements above.
More help on the different commands can be obtained interactively using HELP. for example
\begin{verbatim}
HELP SUBMIT
\end{verbatim}
There are also a number of online documents including a user's guide for the Alpha. To find out the list of 
available documents type  
\begin{verbatim}
DIR RALCCD_DOCUMENTATION
\end{verbatim}
Both text and postscript versions of the files are available. The file names for the user guide are
{RALCCD$\_$DOCUMENTATION:ALPHA$\_$VMS.TXT} and 
\newline
{RALCCD$\_$DOCUMENTATION:ALPHA$\_$VMS.PS}. These can be printed on
one of the printers at RAL or transferred to the users home and printed locally. 


 
\subsection{Running on OSFSERV Unix farm}
The TCP/IP address for OSFSERV is {\bf osfserv.rl.ac.uk}  
(or 130.246.104.2).
Jobs are run on the OSFSERV Unix farm using a script file. Note that the Unix
operating system is case sensitive (unlike VMS on the DEC ALPHA). A script file for running
med103 (med103.scr) is shown next
\begin{verbatim}
# med103.scr:
# QSUB -mb -me
# @$ -eo
# @$ -q long
# @$ -lt 360:00
# @$
#
cd /home/osfserv/ad1/med103
ln -f -s ~/med103/xal100fs.out  fort.11
ln -f -s ~/med103/ial100fs.out  fort.12 
ln -f -s ~/med103/fal100fs.out  fort.13
ln -f -s ~/med103/ral100fs.out  fort.15
f77 -o med103  -r8  -align dcommons med103.for  
med103  < ~/med103/al100fs.dat > ~/med103/oal100fs.out  
\end{verbatim}

The first line is a comment line. The next line
sends a mail at beginning and end of request execution. 
This can then be checked using the command {\bf elm} for example.
The next  line directs all error messages that are normally appended to the standard error file
({\bf stderr}) to the standard output file ({\bf stdout}). The next lines queues the request in the {\bf long} queue. 
Other currently available queues are {\bf short, medium} and {\bf large} queues. The next lines set a limit
of 360 minutes and the following two lines are comments. 

The line beginning with {\bf cd} changes directory to where the med103 files are kept.
You should change
this to point to the directory where you keep your files. 
The next line links the file {\bf xal100fs.out} to fortran stream 11 and the linking to files 
is repeated for stream 12, 13 and 15.
Stream 11 ({\bf xal100fs.out}) contains the X-ray laser output in a
form suitable for the graphic package {\bf xrlrec3} or {\bf xrlcol3}.
Stream 12 ({\bf ial100fs.out})
contains the ionization data in a form suitable for the {\bf ion3} package.
Stream 13 ({\bf fal100fs.out}) contains the hydrodynamic
variables in a form suitable for the graphics package {\bf flip3}.
Streams 15 is used for the restart feature.
The code dumps all variables specifying the target on stream 15 ({\bf ral100fs.out}) (every time
it produces an output on stream 6).
When restarting a previous run the code reads the necessary variables from stream 15.

Under Unix if a file already exists and a new file is created with the same name, the new file
will override the old one. The old data is therefore lost. 
This is unlike VMS where the new file is given a different version number from the old, thus keeping the old data.

The line beginning with {\bf f77} compiles and links the file {\bf med103.for} in double precision
and creates the executable file {\bf med103}. 
If you do not want to recompile the code every time you run it,
you can keep the executable file med103 by putting a comment ({\#})
in front of {\bf f77}.
The final line runs {\bf med103} and reads the standard fortran input 
(stream 5) from {\bf al100fs.dat} and writes the standard output (stream 6) in file {\bf oal100fs.out}. 

Batch jobs are submited to the  Network Queuing System (NQS) using the standard 
command {\bf qsub}.
To run the above script file as a batch job just type
\begin{verbatim}
qsub med103.scr
\end{verbatim}
Almost immediately you will be sent a message.
\begin{verbatim}
Request 18821.osf01.cc.rl.ac.uk submitted to queue: long.
\end{verbatim}
18821.osf01 is the request-id. 

You can find out about the status of your batch job by typing
\begin{verbatim}
qjob
\end{verbatim}
This also tell tells you on which host your request is running (osf01, osf02, osf03, osf04 or osf05).
If for some reason you decide to cancel your request and assuming your request is queued on osf03.cc.rl.ac.uk
type
\begin{verbatim}
qdel -h osf03 18821.osf01
\end{verbatim}
If the job is already executing you should also specify -k
\begin{verbatim}
qdel -k -h osf03 18821.osf01
\end{verbatim}
The queue long has a maximum time limit of 24 hours.
To allow for batch jobs requiring different resources,
a number of different queues are available.
You can  find out the names, limits etc ...  of the other queues by
typing
\begin{verbatim}
more /usr/local/doc/osfserv/nqs_limits.txt
\end{verbatim}
this will list all the queues and also give
you the maximum amount of CPU time allowed on any one queue.
When the job has finished you will be sent a mail which you can view using {\bf elm} for example. 
The file {\bf med103.scr.o18821} which contains any error messages is also created.
More help on the different commands can be obtained interactively using the command {\bf man}. for example
\begin{verbatim}
man qsub
\end{verbatim}
There are also a number of online documents including a user's guide for OSFSERV. To find out the list of
available documents type
\begin{verbatim}
ls /usr/local/doc/osfserv
\end{verbatim}
Bot text and/or postscript versions of the files are available. The file names for the user guide are
{\bf /usr/local/doc/osfserv/osf1gd.txt} and {\bf /usr/local/doc/osfserv/osf1gd.ps}. These can be printed on
one of the printers at RAL or transferred to the users home and printed locally.
               

\subsection{Running on COLUMBUS}
The TCP/IP address for COLUMBUS is {\bf columbus.rl.ac.uk}
(or 130.246.12.27).
A script file for running
med103 on columbus (med103.scr) is shown next
\begin{verbatim}
# med103.scr:
# @$
cd /home/columbus5/ad1/med103
ln -f -s /home/columbus5/ad1/med103/xal100fs.out  fort.11
ln -f -s /home/columbus5/ad1/med103/ial100fs.out  fort.12
ln -f -s /home/columbus5/ad1/med103/fal100fs.out  fort.13
ln -f -s /home/columbus5/ad1/med103/ral100fs.out  fort.15
f77 -o med103  -r8  -fast -align dcommons med103.for
med103  < /home/columbus5/ad1/med103/al100fs.dat > oal100fs.out
\end{verbatim}

This is similar to the script file used in osfserv (see section 2.3 for an explanation).
Batch jobs are submited to the  Network Queuing System (NQS) using the standard
command {\bf qsub}.
To run the above script file as a batch job just type
\begin{verbatim}
qsub -q e2 med103.scr
\end{verbatim}
Notice the queue name (e2) is now given in the command line instead of the
script file. 

Almost immediately you will be sent a message.
\begin{verbatim}
Request 16104 is submitted to queue e2.
\end{verbatim}
16104 is the request-id.

You can find out about the status of your batch job by typing
\begin{verbatim}
qstat
\end{verbatim}
If for some reason you decide to cancel your request 
type
\begin{verbatim}
qdel 16104
\end{verbatim}
If the job is already executing you should also specify -k
\begin{verbatim}
qdel -k 16104
\end{verbatim}
The queue {\bf e2} has a maximum time limit of 6 hours.
To allow for batch jobs requiring different resources,
a number of different queues are available.
You can  find out the names, limits etc ...  of the other queues by
typing
\begin{verbatim}
more /usr/local/doc/columbus/batch_queues.txt
\end{verbatim}
this will list all the queues and also give
you the maximum amount of CPU time allowed on any one queue.
When the job finishes, any error messages are written to 
the file med103.scr.e.
More help on the different commands can be obtained interactively using the {\bf man} command. for example
\begin{verbatim}
man qsub
\end{verbatim}
There are also a number of online documents. To find out the list of
available documents type
\begin{verbatim}
ls /usr/local/doc/columbus
\end{verbatim}
                                                                                          
\newpage
\section{Changing the Input Data File}
This is the most important part of the process by which you may edit
the input data file in order to create the conditions you desire.
MED103 reads in data using NAMELIST.
The NAMELIST is a non-standard FORTRAN extension that works on COLUMBUS, 
OSFSERV, the DEC ALPHA and other machines (although the control
character may change from a dollar to something else). 
A typical data file may look like this (GE1C.DAT)

\begin{verbatim}
GE TARGET (NE-LIKE SCHEME) MODELLED IN planar GEOMETRY                          
100 mic wide (99.5 micron ,0.5 micron Ge)  anabs=0.20                           
I=2.0e13W/cm2  gaussian  ,1.06 micron,Ideal Gas                                 
************************************************************************        
 $NEWRUN                                                                        
 XAMDA1=1.06E-6, GAUSS=1.0, PMAX=2.0E17, ANPULS=1.0,                            
 PLENTH=0.36E-09,  pmult=1.389, toff=1.0e-06,                                   
  MESH=40,   RINI=99.5E-6,    RHOGAS=5323., RF1=0.4232,                         
  XZ=32.0,    XMASS=72.61,   FNE=1.0,                                           
    ZGLAS=10., DRGLAS=0.250E-06, ROGLAS=5323.,  RF2=0.9999,                     
    XZ2=32.,   XMASS2=72.61,   FNE2=1.0,                                        
      ZPLAS=20., DRPLAS=0.250E-06, ROPLAS=5323.,  RF3=0.9181,                   
      XZ1=32.,   XMASS1=72.61,   HYDROG=0.0,                                    
 NGEOM=1, PIQ(55)=0.0, TEINI=300.,TIINI=300.,                                   
 NLRES=f,  NPRNT=-100,TSTOP=1.00e-09,  NRUN=999999, 
 NP3=1,     NLEMP=F,                                                            
 NLBRMS=T,  FLIMIT=10.0,                                                        
 NLCRI1=T,  ANABS=0.20,   FHOT=0.0,       FTHOT=-1.0,    RHOT=1.0,              
 NLPFE=f,    state=0.0, saha=2.0,                                
   ICXRL=1,  ISTAGE=4,                                                          
   IFRSTA=1, ILOSTA=09,   IHISTA=11,                                            
   IDFLG=0,  ROPMUL=1.0,    ITBFLG=0,    ISTFLG=0,                              
   idrflg=1, drmul=1.0, IPUFLG=0,                                               
     NLMAX=2,  FL(1)=0.10e00,fl(2)=0.30,                                          
   ak1=1., ak2=1., ak3=1., ak4=1., ak8=1., ak9=1.,
 $END                                                                           
\end{verbatim}
There is a space in the first column of
this file before the \$, this is necessary so don't remove it.
The four lines written before the \$ have no effect and are thus like a
comment space. These lines do, however, get printed out by the code so
they may be used to identify any particular run
(assuming you change them). These lines are {\bf necessary} as the code
expects four lines of something before it starts to read the NAMELIST
file. The first of these four lines is also included in the titles
given to the graphics output packages so if each run is sensibly
labelled it should be easier to reconcile all the MED103 output
you will inevitably generate.
Going through this NAMELIST file will provide
a user with just about all they need to know about controlling
MED103. The order in which the data appears in the NAMELIST
is not significant.
Also if a constant is not declared in the NAMELIST then it should
default to some reasonably sensible value. This is why most input
data files do not have all the control
variables shown in the following list.
\newline 
{\bf Note: } When editing this file make sure all lines (except the last)
have a comma at the end.
\newline 
Below the letters
C,I,R and L refer to character, integer, real and logical constants
respectively. D denotes the default value. The default value
may stop the code from falling over but it will not necessarily give
sensible results for your problem.

\subsection{Explanation of the input parameters}
\subsubsection{\bf Geometry}
\begin{itemize}
\item {\bf NGEOM (I) D=3}
This variable sets the
geometry of the problem; 1 is planar, 2 is cylindrical and 3 is
spherical. Note: the X-ray laser calculations will at present
only work in cylindrical and planar geometries.
\end{itemize}
\subsubsection{\bf  Laser}
\begin{itemize}
\item {\bf XAMDA1 (R) D=10.0E-6}
laser wavelength (m).
\item {\bf GAUSS (R) D=0.0}
This sets the laser pulse shape. 
\newline GAUSS=1.0
gives a gaussian pulse shape. A single Gaussian pulse is given  
by 
$$P(t)=PMAX\times exp(-((t-PMULT*PLENTH)/PLENTH)^2)$$
\newline GAUSS=--1.0 gives a triangular pulse given by 
$$P(t)=PMAX\times t/PLENTH (for 0 \leq t \leq PLENTH)$$
and 
$$P(t)=PMAX\times (TOFF-t)/(TOFF-PLENTH) (for PLENTH \leq t \leq TOFF)$$  
\newline GAUSS=--2.0 gives a triangular pulse with a flat top (For a pecification see Fig. 1 Appendix C). 
\newline GAUSS=0.0 gives an isentropic pulse.
For a specification of the
isentropic pulse see \cite{cpc} or look in subroutine laser.
\item {\bf TON (R) D=0.0}
This defines the turn-on time of the laser (in s).
This is normally set to 0.0. 
\item {\bf TOFF (R) D=0.0}
This defines the turn-off time of the laser (in s).
This is normally set to a large value such as 1.0. This does not
mean that you will run for a simulation time of 1 second as other
control factors will stop the run well before this.
\item {\bf PMAX (R) D=0.0}
This is the peak power of the laser in units
of W m$^{-2}$ for planar targets, W m$^{-1}$ rad$^{-1}$ for cylindrical
targets and W ster$^{-1}$ for spherical targets. Note: The summary print-out
also includes a factor called PEQUIV which shows the equivalent
surface irradiance in W cm$^{-2}$ for any of the three geometries.
In this way one can check that, for example, a planar experiment
modelled in cylindrical geometry has the right surface irradiance.
\item {\bf PLENTH (R) D=0.0} In the case of one Gaussian pulse,
this is the time from the peak of the laser
pulse to the value at which the laser is only $e^{-1}$ of its maximum
value (in s). This may be approximated to PLENTH=0.6$\times$FWHM.
\item {\bf PMULT (R) D=0.0} In the case of one Gaussian pulse,
this is the number of PLENTH times at which
the run starts before the peak of the pulse. The larger the number the
slower the turn-on of the laser,
this helps avoid shocks at the beginning.
However, the larger the number, the
longer the code will have to run and the more CPU time it will use.
\item {\bf ANPULS (R) D=0.0} When using Gaussian pulses (GAUSS=1.), this sets 
the number of Gaussian pulses. Set this to 1.0, 2.0, 3.0, 4.0 or 5.0. 
Each of the additional 4 pulses is determined by specifying the peak power,
width and time of peak, in a way similar to specifying (PMAX, PLENTH and PMULT) for the first
pulse. The variable which are used for this purpose are shown in Table 1, Appendix C,  and an example input data 
file which uses two pulses can be seen in 
AL100FS.DAT in this manual.
\newline
ANPULS is also used to specify 2 pulses when  GAUSS=-2., 
The variables needed to specify the second pulse in thi case, are given in italic in Fig 1 of Appendix C.
\end{itemize}
\vspace*{5mm}


\subsubsection{\bf Numerical and Target Design}
\newline
MED103 was previously designed to model three layer ICF targets
comprising the inner gas
fill, the middle glass shell and an outer plastic shell. The code was
modified to allow the user to control the contents of the inner gas fill
(referenced as neon). In practise this meant that one could decide that
the proportion of `neon' was, say, 100\% and that the `neon' should have
an atomic charge number of, for example, 79.0 (i.\ e.\ gold). This meant
that the inner fill was solid gold. The code has now been modified to
allow the user to alter not only the inner gas fill but to overwrite the
`glass' and `plastic' constituents of the second and third shell. In
this way MED103 now allows the user complete control over a three layer
target. Some of the names given to the desired quantities will seem odd.
This is because the modifications to MED103 have been made with the
intent of making the code upwardly
compatible with older versions (i.\ e.\
MED103 should run with an old data file with no modifications and
produce the same results, except perhaps where a correction or
modification to the physics has been made. Note however that the
wavelength is now called XAMDA1, not LAMDA1).
\begin{itemize}
\item {\bf MESH (I) D=40}
This controls the total number of cells used in ALL THREE
layers. This number should not exceed {\bf 200}, but this can easily
be changed in the code if necessary. The larger this number is,
the more accurately the layer will be modelled but the longer the
code will take to run. A typical value for MESH is 60.
\item {\bf ZGLAS (R) D=0.0}
This controls the number of cells in the second layer or
`glass' layer. Note: this number is a real. If no second layer is
desired then set ZGLAS=0.0
\item {\bf ZPLAS (R) D=0.0}
This controls the number of cells in the third layer or
`plastic' layer. Note: this number is a real. If no third layer is
desired then set ZPLAS=0.0
\item {\bf RINI (R) D=4.8E-4}
This sets the thickness of the inner layer or `gas' fill(m). Each cell
will then be RINI/(MESH-ZGLAS-ZPLAS) thick, e.\ g.\ if RINI=10.0E-6
and MESH=60, ZGLAS=20.0 and ZPLAS=20.0 then the cell thickness in
the inner layer will be 0.5E-6 meters. The setting of this value
becomes more complicated with the use of the arithmetic gridding factors
(see later parameters and appendix A).
\item {\bf DRGLAS (R) D=0.0}
This sets the thickness of the second layer(m). The thickness of the
individual zones is then DRGLAS/ZGLAS. If no second layer is desired
then specify DRGLAS=0.0. (Note: if ZGLAS=0.0 then DRGLAS is set to 0.0)
\item {\bf DRPLAS (R) D=0.0}
This sets the thickness of the third layer(m). The thickness of the
individual zones is then DRPLAS/ZPLAS. If no third layer is desired then
specify DRPLAS=0.0. (Note: if ZPLAS=0.0 then DRPLAS is set to 0.0)
\item {\bf RHOGAS (R) D=124.0}
This is the density of the inner layer or `gas fill'. Units are
kgm$^{-3}$.
\item {\bf ROGLAS (R) D=0.0}
This is the density of the middle layer or `glass shell'. Units are
kgm$^{-3}$.
\item {\bf ROPLAS (R) D=0.0}
This is the density of the outer layer or `plastic shell'. Units are
kgm$^{-3}$.
\item {\bf XZ (R) D=1.0}
This is the atomic number of the inner layer or `gas fill' material.
\item {\bf XZ2 (R) D=14.0}
This is the atomic number of the middle layer or `glass shell' material.
\item {\bf XZ1 (R) D=6.0}
This is the atomic number of the outer layer  or `plastic
shell' material.
\item {\bf XMASS (R) D=1.}
This is the atomic mass number of the inner layer or `gas fill' material.
\item {\bf XMASS2 (R) D=28.0855}
This is the atomic mass number of the middle layer or `glass
shell' material.
\item {\bf XMASS1 (R) D=12.011}
This is the atomic mass number of the outer layer
or `plastic shell' material.
\item {\bf FNE (R) D=0.0}
This is the fraction of the material specified by XZ, XMASS and RHOGAS
that is used in the inner layer. The remaining fraction is divided
equally between deuterium and tritium. This is really only useful for
modelling ICF targets, to fill the inner layer entirely with a specified
material set FNE=1.0.
\item{\bf FNE2 (R) D=0.0}
This is the fraction of the material specified by XZ2, XMASS2 and ROGLAS
that is used in the middle layer. The remaining fraction is set to be
silicon. Again, this is left--over from modelling ICF targets. To get
control specify FNE2=1.0.
\item{\bf HYDROG (R) D=0.0}
This controls the fraction of hydrogen in the plastic or outer shell. If
HYDROG=0.0 then the outer layer is composed entirely of the material
specified by XZ1, XMASS1 and ROPLAS. Anything greater than 0.0 will
result in a plastic of some form. For example, to obtain CH$_2$ use
HYDROG=0.67.
 
As stated earlier MED103 has been altered to allow the user control
over the constituent materials in all three layers. If a mixed layer is
desired then the atomic numbers and mass numbers should be averaged.
From the point of
view of modelling the hydrodynamics, averaging the atomic and mass
numbers is acceptable, providing the real
density is used for the material.
For example, to have a plastic layer as the inner
target layer corresponding to CH then the charge number would be 3.5 and
the mass number would be 6.5 with a density of 1000 kgm$^{-3}$.
Note: this does not make sense for X-ray laser calculations
(see below).
\item {\bf RF1 (R) D=0.99999}
This controls the arithmetic gridding in the `gas fill' or inner layer.
\item {\bf RF2 (R) D=0.99999}
This controls the arithmetic gridding in the `glass' or middle
layer.
\item {\bf RF3 (R) D=0.99999}
This controls the arithmetic gridding in the `plastic' or outer layer.

For a definition of RF1, RF2 and RF3 see appendix A.
A simple fortran programme which calculates the RF value given the total thickness,
the thickness of the first cell and the number of meshes is given in appendix B.
 
When using planar geometries it is possible to specify whether the left
and right boundaries of the mesh are fixed or free. This is controlled by
the variable PIQ(55). Bear in mind that MED103 treats the laser as
travelling from right to left (from a high cell number towards cell
number 1). 
\item {\bf PIQ(55) (R) D=0.0}
PIQ(55) is set to 0.0 by default and if NGEOM$\not=$1 then the
left boundary is fixed with PIQ(55) controlling the right boundary.
A thin foil, for example, would be best modelled with PIQ(55)=2.0.
The various combinations are:
\begin{itemize}
\begin{itemize}
\item{0.0} Left boundary fixed. Right boundary free.
\item{1.0} Left boundary fixed. Right boundary fixed.
\item{2.0} Left boundary free. Right boundary free.
\item{3.0} Left boundary free. Right boundary fixed.
\end{itemize}
\end{itemize}
\end{itemize}
 

\subsubsection{\bf Physics}
\begin{itemize}
\item {\bf TEINI (R) D=1.0E3}
This is the initial temperature of the electrons, normally set to
a value of, typically, 300 K.
It is now possible to specify a different initial electron temperatures for each layer if desired (see table 2 in Appendix C).
\item {\bf TIINI (R) D=1.0E3}
This is the initial temperature of the ions, normally set to
the same value as TEINI, typically, 300K.
It is now possible to specify a different initial ion temperatures in each layer if desired (see table 2 in Appendix C).
\item {\bf ANABS (R) D=0.0}
This is the amount of laser energy that having reached critical
density (i.\ e.\ having been to some extent depleted by inverse
bremsstrahlung absorption) is deposited in the plasma by
resonance absorption. Typically 10\%-20\% (e.\ g.\ ANABS=0.2).
Note: this only works if NLCRI1=.TRUE.
\item {\bf FHOT (R) D=0.0}
This is the fraction of laser energy absorbed by
resonance absorption that gets converted to hot electrons, typically
0.1. This is not the same as setting the amount of laser energy
deposited by resonance absorption.
\item {\bf FTHOT (R) D=0.0}
This determines the way in which the code calculates the
hot electron temperature $T_H$.
\newline
If FHOT$>$0.0 then $T_H=$FTHOT$\times T_{ec}$ where $T_{ec}$ is
the electron temperature at critical density. 
\newline
If FHOT$<$0.0 then
$T_H=-$FTHOT$\times f(I\lambda^2)$ where $f(I\lambda^2)$ is the
LASL compilations of $T_H$ versus $I\lambda^2$. 
\newline
The transport of energy 
of hot electrons is handled by splitting the Maxwellian distribution 
into 10 groups and performing a dE/dx calculation for each group. see subroutine ABSORB for details.
\item {\bf RHOT (R) D=0.0}
This controls the transport of hot electron energy after the electrons
have made a single pass through the target. If RHOT=1.0 the energy
remaining at this time is deposited throughout the target uniformly.
If RHOT=0.0 then the remaining energy is lost. Typical value of
RHOT is 1.0.
\item {\bf FLIMIT (R) D=0.0}
This is the control of the flux limiter where
$${\rm FLIMIT}= {{\rm classical\ free\ streaming\ limit}
\over {\rm desired\ flux\ limit}}$$
A typical value of FLIMIT is 10.0 (i.\ e.\ the desired flux limit is 0.1)
\item {\bf PIQ(27) (R) D=0.0}
It is possible to vary the thermal conductivity of the electrons, $K_e$,
using
$$K_e = K(Spitzer) \times (1.0 + {\rm PIQ(27)})$$
As all elements of the PIQ array default to zero the conductivity will
be the Spitzer value by default.
\item {\bf NLOMT3(10) (L) D=F}
This controls whether non local
electron heat conduction
is used instead of flux limited Spitzer formulation (above). For planar geometry
the formulation by Luciani et al\cite{Luciani} is used. This is generalized to cylindrical and spherical geometry
using an approximate (and less rigorous)  method.
\item {\bf STATE (R) D=0.0}
The ions always have a perfect gas equation of state
regardless of the value of NLPFI (see next section).
If NLPFE=T then the electrons also
use the perfect gas equation of state. If NLPFE=F then the equation
of state is decided by the value given to STATE. Typically, STATE=3.0
is reasonable when using SAHA=1.0. If NLPFE=T, this will save CPU time but the physics may
not be strictly applicable to your specific situation. The values of
STATE are
\begin{itemize}
\begin{itemize}
\item {0.0} Fermi Dirac
\item {1.0} Thomas Fermi
\item {2.0} Thomas Fermi with quantum corrections
\item {3.0} Thomas Fermi with modified corrections to
give correct solid density.
\end{itemize}
\end{itemize}
\item {\bf SAHA (R) D=0.0}
SAHA determines whether or not to calculate ionization equilibrium.
\newline
If SAHA=0.0, the material is assumed to be fully ionized.
\newline
If SAHA=1.0
then the ionization is calculated from Saha's equation (Z $\le$ 18) or
the Thomas-Fermi model for higher Z values.
\newline
If SAHA=2.0 then the ionization is calculated from the time-dependent
solution of the average atom rate equations.
\newline
{\bf Note:} If the X-ray laser
calculations are turned on, then the ionization balance must be calculated
using SAHA=2.0.
\item {\bf PONDF (R) D=0.0}
This controls whether or not the ponderomotive force of the laser
is included in the momentum equation. PONDF=1.0 includes the
ponderomotive force. This uses a very crude prescription for the ponderomotive force and should be
used with caution. The ponderomotive force is calculated in the WKB approximation only and the code
calculations will be illustrative rather than quantitative (see subroutine FORMP).
\end{itemize}
\vspace*{5mm}

\subsubsection{\bf Physics: Logical Switches}
\newline
This section details the logical switches that control whether
certain pieces of physics are to be included in the calculation.
If {\bf switch}=.TRUE. (or T) the part is included,
if {\bf switch}=.FALSE. (or F) it is not.
\begin{itemize}
\item {\bf NLTNL (L) D=F}
This controls whether
tunnel ionization 
is included in the average atom rates.
\item {\bf NLABS (L) D=T}
This controls whether
the absorption of laser radiation by inverse bremsstrahlung
is included in the calculation.
\item {\bf NLHF (L) D=F}
This controls whether a high field correction is applied
to the absorption of laser radiation by inverse bremsstrahlung.
\item {\bf NLBRMS (L) D=T}
This controls whether
radiation from bremsstrahlung in the plasma
is included in the calculation.
\item {\bf NLBURN (L) D=T}
This controls whether
burning of fusion fuel
is included in the calculation.
\item {\bf NLCRI1 (L) D=T}
This controls whether
any laser energy is dumped at critical density
in the calculation.
\item {\bf NLDEPO (L) D=T}
This controls whether
any energy from fusion products
is included in the calculation.
\item {\bf NLECON (L) D=T}
This controls whether
electron heat conduction
is included in the calculation.
\item {\bf NLICON (L) D=T}
This controls whether
ion heat conduction
is included in the calculation.
\item {\bf NLFUSE (L) D=T}
This controls whether
fusion from D-D, D-T and D-He3 (if present)
is included in the calculation.
\item {\bf NLMOVE (L) D=T}
This controls whether
fluid motion
is included in the calculation.
\item {\bf NLPFE (L) D=T}
This controls whether
the equation of state for electrons is the perfect gas EOS.
\item {\bf NLPFI (L) D=T}
This controls whether
the equation of state for ions is the perfect gas EOS.
\item {\bf NLX (L) D=T}
This controls whether
ion-electron collisional relaxation
is included in the calculation.
\end{itemize}
\vspace*{5mm}



\subsubsection{\bf Run Control}
\begin{itemize}
\item {\bf NLRES (L) D=F} This switch specifies if the code should restart a previous run.
If this switch is set to true (T), the code will read the previous simulation dump from fortran stream 15
and carry on from there.
\item {\bf TSTOP (R) D=1.0E-6}
This is the time in the simulation at which the calculation stops
(units are seconds).
\item {\bf NRUN (I) D=100}
This is the maximum number of time steps that the code is allowed to run
for. Set this to something large (e.\ g.\ 80000) so that the code will
always terminate because of TSTOP rather than NRUN --- this makes it
easier to terminate the run exactly when you want.
\item {\bf NPRNT (R) D=100.0}
This is the number of time steps between printer (and restart and graphics
files) dumps. If NPRNT
is negative then this number becomes the number of picoseconds between
successive printer (and restart and graphics files) output. For example, to produce
a printout every 100 femtoseconds use NPRTNT=-0.1.
\item {\bf NLEMP (L) D=T}
This controls whether you receive reams of diagnostic print-out when the
code falls over. Probably best set to .FALSE.
\item {\bf NP3 (I) D=(MESH/20)}
This how often the range of cells is sampled in order to provide output
for the printer and graphics dumps, i.\ e.\ it is part of a construct
\begin{verbatim}
      DO 10 I=1,MESH,NP3
        WRITE ...
   10 CONTINUE
\end{verbatim}
Set NP3=1.
\end{itemize}
\vspace*{5mm}

\subsubsection{\bf Code Control}
\newline
This section lists some of the parameters available to the user for
altering some of the numerical parameters in the code. Generally,
these values should not be altered unless a specific problem
is encountered, it is best to seek advice.
\begin{itemize}
\item {\bf DELTAT (R) D=1.0E-18}
This is the initial timestep, the code will automatically choose
its own value for this after a few time steps. Something small, typically
$10^{-18}$s (the default) is recommended.
\item {\bf DTMAX (R) D=1.E-09}
This is the maximum time step that the code is allowed to use.
\item {\bf AK0 (R) D=5.0}
The is the maximum allowable ratio of successive time steps. Set AK0
to something large (e.\ g.\ 100.0) if message `time centering is damaged'
appears.
\item {\bf AK1, AK2, AK3, AK4, AK5 (R) D=0.25,0.25,0.25,0.25,0.0}
These numbers control the choice of timestep by the code. Refer to routine
TIMSTP. AK1 controls the CFL condition:$\Delta t<\Delta r/c_s$.
AK2--4 control the range of density,
ion temperature and electron temperature
respectively, AK5 is not used at present.

\item {\bf AK8, AK9,  (R) D=0.25,0.25}
These numbers also control the choice of timestep by the code when usind the average atom model (SAHA=2.). 
Refer to routine
TIMSTP. AK9 was introduced to control the change in average ionization state
(useful when tunnel ionization is switched on). AK8 was introduced to control the change in 
energy input in a given computational cell 
(useful when simulating high peak power short pulses).  
Note: AK6 and AK7 are not used at present.

\item {\bf NLITE (L) D=T}
This is a logical switch that controls iterations on $T_i$, $T_e$ and $U$
(the internal energy). Since the thermal conductivity is non-linear
it is best to iterate. Calculations which have electron thermal conduction
switched off could set NLITE=F.
\item {\bf DUMAX, DTEMAX, DTIMAX (R) D=0.1,0.1,0.1}
The maximum fractional change in $U$, $T_e$ and $T_i$ which represent
acceptable convergence.
\item {\bf NITMAX (I) D=5}
The maximum number of iterations before code stops due to the iterations
failing to converge.
\end{itemize}


\subsubsection{\bf X-Ray Laser Calculations}
\newline
One of the major new additions in MED103 and MED101 are subroutines for calculating
X-ray laser gain in collisional \cite{djaoui2} as well as recombining plasmas \cite{sjr}.
The recombining option  is suitable for the three main single active
electron atom/ion lasing schemes (i.\ e.\ H-like, Li-like and Na-like) and the collisional
option is only implemented for the Ne-like scheme.
An option for
modelling photopumping is included.
The x-ray laser subroutines are an integral part of MED103 but are switched
off by default so that users only interested in modelling
hydrodynamics may do so by setting (ICXRL=0).

 
When X-ray laser gain is calculated the non-LTE time dependent average atom model should be used to
calculate the fractions of different ion stages within the plasma by setting SAHA=2.
The X-ray laser data output includes
gain as a function of radius for every time frame dumped, refraction
of the X-ray beam (based upon the width and length of the X-ray laser
target which is supplied by the user)
and a spatially integrated measure of
gain that may be compared directly with the experimental results of X-ray
laser experiments \cite{sjr}.
All of this data is printed out on the standard MED103
printout but is also written to data files for interpretation by the
graphics packages. The X-ray laser code will
run in cylindrical and planar
geometries.
 
There is also an option to add a scaling factor to
the opacity of the resonance lines
when using the average atom model. This is by default set to zero (i.\ e.\
no absorption of the resonance line is included),
if the scaling factor
flag is set equal to 1 then a variable scaling factor may be used
(this has been included to investigate the best agreement with
recombination experimental results).
 
As a sub-set of the X-ray laser calculations the spatially
dependent ion densities
of selected ionic species may also be output. Only the ground state
number densities are calculated (typically 90--99\% of the ions are
in their ground state).
An option is supplied to calculate
exactly what percentage of all of the ion states is in the
ground state.
 
All of these options are
controlled by various switches and variables now included in the NAMELIST
file. For the Integer type switch set the variable equal to 1 for
the particular calculation to be carried out.
 
{\bf Note:} in X-ray laser calculations pure
atomic numbers and atomic mass numbers (XZ and XMASS  etc.)\
must be used. However the real densities of the target materials
should be used. For example, to model the carbon in a plastic target
in an X-ray laser calculation use
a density of 1000 kgm$^{-3}$ (if it uses a plastic target) rather than
using the higher density appropriate for pure carbon, but
with a charge number of 6 and a mass number 12.
\begin{itemize}
\item {\bf ICXRL (I) D=0}
This is a switch that controls whether the
X-ray laser calculations are to be included.
\item {\bf ISTAGE (I) D=1}
This controls the scheme used in the calculation:
\newline
1 H-like
\newline
2 Li-like
\newline
3 Na-like
\newline
4 Ne-like
\item {\bf DLAMDA (R) D=81.0E-8}
If ISTAGE $\le$ 3, DLAMDA is the wavelength of the main X-ray lasing transition (units are cm).
\item {\bf IFRSTA (I) D=0}
This is a switch that controls whether the
spatially dependent ion densities
are included in the calculation. 
This can only  be carried out at present
when SAHA=2.
\item {\bf IDFLG (I) D=0}
This is a switch that controls whether trapping of the resonance line
is included in the calculation of the average atom level populations.
\item {\bf ROPMUL (R) D=0.0}
This is the scaling factor that alters the escape
factor of the resonance line
in the calculation. Set equal to 0.0 if no trapping is desired (or
set IDFLG=0), set =1.0 if full trapping is to be included. Any other
value chosen for this variable is up to you.
\item {\bf ITBFLG (I) D=1}
\itemitem{0} No thermal band.
\itemitem{1} Forces the population of the highest level to be in LTE.
\item {\bf ISTFLG (I) D=0}
This controls whether motional Stark broadening is to be included in
the calculation (NB: Only data for carbon is at present included).
\itemitem{0} Motional Stark broadening is not included.
\itemitem{1} Motional Stark broadening is included on the Balmer alpha
line.
\item {\bf IPUFLG (I) D=0}
This controls whether photopumping is to be included in the calculation.
\item {\bf TPON (R) D=0.0}
This is the time at which the photopumping is switched on
(units are seconds).
\item {\bf TPOFF (R) D=0.0}
This is the time at which the photopumping is switched off
(units are in seconds).
\item {\bf NLP (I) D=0}
This is the lower level (corresponding to the principal quantum number
of the state in question) with which a coincidence photopumping line
is to be included.
\item {\bf NUP (I) D=0}
This is the upper level (corresponding to the principal quantum number
of the state in question) with which a coincidence photopumping line
is to be included.
\item {\bf RMPD (R) D=0.0}
This is the modal photon density of the pumping radiation.
\item {\bf FLSHORT (R) D=0.1}
This is the short length of the X-ray laser gain medium
(i.\ e.\ the width
of the gain region), units are in cm. (This is the same as FL(1))
\item {\bf FLLONG (R) D=1.0}
This is the long length of the X-ray laser gain medium, units are in cm. (This is the same as FL(2))
 
The sub-set of the X-ray laser calculations that allows the spatially
dependent ion densities of selected ion species in ground state
to be calculated
are controlled by ILOSTA and IHISTA.
These specify the range of ionization
stages that the calculation will use.
\item {\bf ILOSTA (I) D=1}
This is the number of electrons in the most
ionized ion to be considered
(the ion with the least bound electrons). For
example, if ILOSTA=0 then the code will supply data on completely
stripped ion densities, if ILOSTA=1, H-like ions, and so on.
\item {\bf IHISTA (I) D=2}
This is the number of electrons in the least
ionized ion to be considered
(the ion with the most bound electrons). For example,
if ILOSTA=1 and IHISTA=3 then the code will supply the spatially
dependent ion densities of H-like, He-like and Li-like ions.
\item {\bf IGSTAT (I) D=0}
This is a switch that controls whether the total number of ions
in all ionization stages in the ground state, is calculated. This
can be used to check whether the data supplied by the routines
controlled by ILOSTA and IHISTA is reasonably valid. A high percentage
of ions in the ground state means that this data is good.
Note: Because this call calculates all of the possible ground state
ions, it tends to eat up CPU time, so allow extra time if you
include this.
\end{itemize}
\vspace*{5mm}

\newpage 
\section{General Hints}
The following section outlines briefly the steps that a user should
go through in order to set up data for a run.
\begin{itemize}
\item Specify laser wavelength, power (check units and geometry),
pulse shape
and duration. For Gaussian pulses set ANPULS to the number used ($\le$5)
\item Specify the target material/s. Build the target from the inner
layer outwards, i.\ e.\ if only one layer is desired then use the inner
or `gas' fill layer. Check MESH is greater than (ZPLAS+ZGLAS).
Check values of arithmetic gridding factor, if used. Check adjacent
layers do not have a difference of more than a factor of 2 in mass
(this applies when different target
materials are used in different layers).
For example, if layer 1 has a material with a density of $x$ kgm$^{-3}$
then if layer 2 has a material with density $2x$ kgm$^{-3}$ then the cell
size of layer 2 should be half that of layer 1.
 
In specifying the physics the following applies to most
normal situations:
\item Always set RHOT=1.0 and FTHOT=--1.0
\item If you know the absorption in a particular experiment that you are
simulating, set NLABS=F and ANABS=$actual\ absorption$. If you have no
information on the amount of hot electrons generated, set FHOT=0.5.
\item The most useful way to get output over your run is to set
it to print out data between 10-20 times over the duration of the run.
This is most accurately controlled by using NPRNT with a negative value
so that it prints at specific times
rather than using a positive value that
prints out after a specific number of timesteps (the difference between
print-outs then varies), i.\ e.\ NPRNT=INT[(TSTOP*10$^{12}$)/10.0.]
\item Give TIINI and TEINI identical values.
\item When modelling recombination X-ray lasers, generally speaking, use cylindrical
geometry. If the aim is to model a planar experiment then the code may be
used provided the radius of the target is made large enough to model
the plasma expansion as approximately planar. A general rule of thumb
is to have the radius the same size as the width of the planar target to
be modelled. This may, however, produce a plasma that cools too quickly
and incorrectly model the resulting gain. If a larger radius is used the
expansion will be more similar to a planar expansion. If in doubt,
set at least one run to use real planar expansion and use this as the
lower limit of the observed gain.
\end{itemize}
 
\subsection{Common Error Messages}
There are two fairly common error messages:
\begin{itemize}
\item TIME CENTERING DAMAGED means that the ratio of successive time
steps is less than AK0$^{-1}$. The remedy for this is to make AK0 larger,
for example, set AK0=100.0. There is no penalty in run time and only a
small likelihood of a serious numerical error.
\item ITERATIONS FAIL TO CONVERGE VARIABLE $n$ IN CELL $l$. This is
self-explanatory, the variable $n$ corresponds to the quantities:
1=velocity, 2=density (specific volume), 3=$T_i$ and 4=$T_e$. The
approach should be to reduce the value of AK$n$ (e.\ g.\ AK4 if $T_e$
fails to converge) to a smaller value. There will be an increase in
run time as the values of AK$n$ are reduced but the exact amount
is unpredictable.
\item POPULATIONS FAILED TO CONVERGE IN CELL $n$. This will only
be seen in the time dependent ionization part of the code. This normally
happens only at low temperatures whenever 
a cell starts to heat up at the very beginning of a code run. At the start of
the run the system will try to ionize very rapidly; if the timestep
is too large (the timestep is determined by the
hydrodynamics) then the
calculation determining the ionization will fail to converge. If this
only occurs at the start of the code (within 100ps)
then ignore it --- the system will
settle down after a short time. If the message occurs late in the
run then there may be something wrong with the way the run has been
set up.
\item There are messages DENSITIES/TEMPERATURES TOO LOW. These
will only normally be seen if the code has been allowed to run for
extremely long times (e.\ g.\ greater than $10^{-6}$).
\end{itemize}

\newpage
\section{Dealing with the output}
When the job has finished on AXPRL2, OSFSERV or COLUMBUS
(or has failed due to lack of time or space,
or because of an error) all the output files will be sent to
the files specified in the COMMAND or SCRIPT file.
You can either view them on screen using and editor (EDIT on AXPRL2 and
emacs on OSFSERV and COLUMBUS for example) or
print them. you can use the  LPEM command to run a
preprocessor on the output files.
The output can be formatted in portrait or landscape
mode with either one, two or four pages to a sheet of paper. The number
of lines  per page and characters per line can also be altered.

The following pages which show some typical MED103 output on stream 6, obtained
using the input data file below (AL100FS.DAT) where formatted using
\begin{verbatim}
lpem -u2 -ol -f oal100fs.out oal100fs.pem
\end{verbatim}
before being printing using
\begin{verbatim}
lpr oal100fs.pem
\end{verbatim}
                                             
\newline
{\bf Table 1.1}: title page, input data as read from stream 5
and a summary of the data.
The remainder of the output comprises frames for the initial
and later times (determined by the value of NPRNT) similar to tables 1.2 to 1.6
\newline
{\bf Table 1.2}: Hydrodynamic conditions of target at specified times.
All quantities are calculated at the cell centre except the
velocity which is calculated at the cell edge.
For mixtures Z is the average atomic number.
\newline
{\bf Table 1.3}: Laser-related variables across target at specified time.
(Intensity, absorption coefficient etc...)
\newline
{\bf Table 1.4 and 1.5}: non-LTE populations from the average atom model
and corresponding LTE populations
for the same density and temperature.
This is only printed if SAHA=2.
\newline
{\bf Table 1.6}: Ground state ion fractions and densities for some ion stages
specified by ILOSTA and IHISTA. This is only printed if SAHA=2 and IFRSTA=1.

                                      
\newpage
\vspace*{9.15in}
Table 1.1: title page, input data as read from stream 5
and a summary of the data.
%set{counter}{}
\newpage
\vspace*{9.15in}
Table 1.2: Hydrodynamic conditions of target at specified time.
\newpage
\vspace*{9.15in}
Table 1.3: Laser-related variables across target at specified time.
\newpage
\vspace*{9.15in}
Table 1.4: non-LTE populations from the average atom model
in each cell.
\newpage
\vspace*{9.15in}
Table 1.5: LTE populations from the average atom model
in each cell.
\newpage
\vspace*{9.15in}
Table 1.6: Ground state ion fractions and densities for different ion stages
 
\subsection{Using the Graphics Packages}
The packages FLIP3, ION3, XRLREC3 and XRLCOL3  are all available
for interpretation of the files output by MED103 on streams 11, 12 and 13.
\newline
1) FLIP3 --- this plots the hydrodynamic variables (velocity,
density, pressure, electron and ion temperatures,
and average ionization) vs.\ distance
for different times, and also the hydrodynamic variables
(cell edge and centre plus velocity  etc.)\ vs.\ time
for each cell
\newline
2) ION3 --- this plots the ground state number density
of the different ionization stages vs.\ distance at
different times during the interaction.
ION3 also displays the zoning of the run. This can be useful to
ensure that there are enough cells in the regions of interest.
\newline
3) XRLREC3 --- this plots the recombination X-ray laser gain vs.\
distance at different
times, and also the space-integrated gain for different
lines (i.\ e.\ alpha, beta etc.)\ vs.\ time as would
as measured by experiment.
\newline
4) XRLCOL3 --- this plots the Ne-like collisional X-ray laser gain vs.\
distance at different
times, and also the space-integrated gain for different
lines (i.\ e.\ alpha, beta etc.)\ vs.\ time as would
as measured by experiment.

The use of the above packages on AXPRL2 is described next.  
The fortran files for these programmes can be obtained from DISK$\_$USER3:[AD1.PUBLIC]
on machine AXPRL2.RL.AC.UK. These packages can also be run on OSFSERV and COLUMBUS using appropriate
script files.

 
All the graphs produced include six lines of data, (including
a run name) compiled from the input data file recording, detailing
most of the parameters used in that MED103 run.
 
\subsubsection{FLIP3}
FLIP3 reads data from stream 13 and writes to stream 16.
For example, in order to display the hydrodynamic data from the example data file (AL100FS.DAT)
you need a command file such the following.
\begin{verbatim}
!flip3.com
$set default disk_user3:[ad1.med103]
$fortran flip3.for
$link/executable=flip3 flip3,nag$libj06/lib,nagg$gks/lib,nag$library/lib,gks_opt/opt
$assign   week_disk:[public.week.ad1]fal100fs.out for013
$assign   week_disk:[public.week.ad1]fal100fs.ps  for016
\end{verbatim}
first execute the file using
\begin{verbatim}
@FLIP3
\end{verbatim}
then run the executable using
\begin{verbatim}
RUN FLIP3
\end{verbatim}
First you will be asked
\begin{verbatim}
DO YOU WANT TO READ DATA FOR ALL           65 POINTS? Y/N ?
\end{verbatim}
Most times you should enter Y unless you want to read/plot data
for a few cells only. Then you will be asked
\begin{verbatim}
 DO YOU WANT TO
  S  Plot quantities vs SPACE
  T  Plot quantities vs TIME
  X  EXIT
\end{verbatim}
Enter S to plot quantities vs.\ space and T to plot quantities
vs.\ time. When plotting vs.\ space you will be asked
\begin{verbatim}
WHAT DO YOU WANT TO PLOT
   1. HYDRODYNAMIC VELOCITIES
   2. DENSITY
   3. PRESSURE
   4. ELECTRON TEMPERATURE
   5. ION TEMPERATURE
   6. AVERAGE IONIZATION
   7. ELECTRON DENSITY
   0. RETURN TO SPACE-TIME-EXIT OPTION

      Enter choice (0 - 7)
 ?
\end{verbatim}
if you enter 1, 2, 3, 4, 5, 6 or 7  you will be asked
\begin{verbatim}
 DO YOU WANT TO PLOT ALL TIME FRAMES? Y/N
\end{verbatim}
If you answer Y you will be asked
\begin{verbatim}
 DO YOU WANT LOGS ON THE Y-AXIS WHERE POSSIBLE? Y/N
 
 DO YOU WANT THE OVERALL MAXIMA FOR EVERY TIME FRAME? Y/N
\end{verbatim}
Plotting the frames with the overall maxima allows comparisons
to be made between the different time frames. If you answer
N to this prompt each time frame will have local X and
Y maxima (which will be different for every frame).
 
If you had entered N to the PLOT ALL TIME FRAME
question you would be asked 
\begin{verbatim}
DO YOU WANT TO SPECIFY FIRST AND LAST TIME(S) TO PLOT? Y/N
\end{verbatim}
If you had entered N you will asked 
several questions about each
time frame. These prompts
are self-explanatory, if somewhat tiresome especially when there are many time frames.
It is therefore advisable to specify the first and last time you want to plot.

\newline 
If you had opted to plot against time by entering T (instead of S above)  you will be prompted
\begin{verbatim}
DO YOU WANT to write data(time) in a file? Y/N
\end{verbatim}
This is useful if you want to export the data file to another package for plotting
(for example the radius as a function of time of all cells of an ICF target can be generated this way).
Then you will be asked
\begin{verbatim}
WHAT DO YOU WANT TO PLOT?
   1. CELL EDGES
   2. HYDRODYNAMIC VELOCITIES
   3. CELL CENTRES
   4. DENSITY
   5. PRESSURE
   6. ELECTRON TEMPERATURE
   7. ION TEMPERATURE
   8. AVERAGE IONIZATION
   9. ELECTRON DENSITY
   0. RETURN TO SPACE-TIME-EXIT OPTION

      ENTER CHOICE (0 - 9)
 ?
\end{verbatim}
You are then given the option of only plotting part of the
range on the X(time) or Y axis. You are then asked
\begin{verbatim}
 DO YOU WANT TO PLOT ALL CELLS? Y/N
\end{verbatim}
Any of the following types of responses are valid
\begin{verbatim}
ALL
1 to 10
1,3,12 to 25,45
\end{verbatim}
Except for the cell edges, velocities and cell centres it is not
advisable to plot all the cells as the results are usually
totally confusing.

Some example output from FLIP3 (and FAL100FS.OUT) are shown in Figs 1 and 2. Figure 1 a plot obtained by
opting to plot log electron temperature vs. distance for all times not using the overall maxima for 
every time frame. Figure 2 shows the position of all cell centres plotted vs. time.
 
\subsubsection{ION3}
ION3 reads data from stream 12
and writes to stream 15. 
Therefore if your input data is in
IAL100FS.OUT (as it would be if you used the COMMAND file in section 'RUNNING ON THE DEC ALPHA')
and you want to call your output graphics file IAL100FS.PS,
you need a command file such the following.
\begin{verbatim}
!ion3.com
$set default disk_user3:[ad1.med103]
$fortran ion3.for
$link/executable=ion3 ion3,nag$libj06/lib,nagg$gks/lib,nag$library/lib,gks_opt/opt
$assign week_disk:[public.week.ad1]ial100fs.out for012
$assign week_disk:[public.week.ad1]ial100fs.ps  for015
\end{verbatim}
first execute the file using
\begin{verbatim}
@ION3
\end{verbatim}
then run the executable using
\begin{verbatim}
RUN ION3
\end{verbatim}
First you will be asked
\begin{verbatim}
DO YOU WANT TO PLOT ALL           65 POINTS? Y/N
\end{verbatim}
Most times you should enter y unless you want to read/plot data
for a few cells only. Then you will be asked
\begin{verbatim}
RAL GKS 1.38 ALPHA/VMS version
Enter minimum value of Y::(Default=1.40000E+01):
\end{verbatim}
All prompts in ION3 have have sensible defaults and if you hit the ENTER key after each query you should 
obtain useful output.

Figure 3 shows one of the plots of ion density vs. distance (for 4 ion stages). The position of the cell edges
are shown below the X-axis, allowing the zoning of the run to be monitered.


\subsubsection{XRLREC3}
XRLREC3 plots the x-ray laser gain data for the recombination case (ISTAGE=1, 2 or 3). It reads data from stream 11
and writes to stream 14. 
Therefore if your input data is in
XFPAP2.OUT 
and you want to call your output graphics file XFPAP2.PS,
you need a command file such the following.
\begin{verbatim}
!xrlrec3.com
$set default disk_user3:[ad1.med103]
$fortran/nooptimize xrlrec3.for
$link/executable=xrlrec3 xrlrec3,nag$libj06/lib,nagg$gks/lib,nag$library/lib,gks_opt/opt
$assign   week_disk:[public.week.ad1]xfpap2.out for011
$assign   week_disk:[public.week.ad1]xfpap2.ps  for014
\end{verbatim}
first execute the file using
\begin{verbatim}
@XRLREC3
\end{verbatim}
then run the executable using
\begin{verbatim}
RUN XRLREC3
\end{verbatim}
First you will be asked
\begin{verbatim}
DO YOU WANT TO USE ABSOLUTE TIME OR TIME RELATIVE
TO THE PEAK OF THE LASER PULSE? A/R
\end{verbatim}
Enter A or R. You will then be asked Y/N questions.
\begin{verbatim}
 ncell           50 ngain            3
RAL GKS 1.38 ALPHA/VMS version
DO YOU WANT TO PLOT GAIN VS RADIUS? Y/N
\end{verbatim}
You will also be given the option of plotting the alpha, beta
etc.\ gain vs.\. time on separate plots  and then of plotting
all the gains on one plot.

Figure 4 shows some of the graphs obtained by plotting x-ray laser gain versus distance. Figure 5 shows 
the efective gains (spatially integrated) for Alpha, Beta and Gamma lines vs. time.


\subsubsection{XRLCOL3}
XRLCOL3 plots the x-ray laser gain data for the Ne-like collisional case (ISTAGE=4). It reads data from stream 11
and writes to stream 14.
Therefore if your input data is in
XGE1C.OUT
and you want to call your output graphics file XGE1C.PS,
you need a command file such the following.
\begin{verbatim}
!xrlrec3.com
$set default disk_user3:[ad1.med103]
$fortran/nooptimize xrlcol3.for
$link/executable=xrlcol3 xrlcol3,nag$libj06/lib,nagg$gks/lib,nag$library/lib,gks_opt/opt
$assign   week_disk:[public.week.ad1]xge1c.out for011
$assign   week_disk:[public.week.ad1]xge1c.ps  for014
\end{verbatim}
first execute the file using
\begin{verbatim}
@XRLCOL3
\end{verbatim}
then run the executable using
\begin{verbatim}
RUN XRLCOL3
\end{verbatim}
First you will be asked
\begin{verbatim}
DO YOU WANT TO USE ABSOLUTE TIME OR TIME RELATIVE
TO THE PEAK OF THE LASER PULSE? A/R
\end{verbatim}
Enter A or R. You will then be asked Y/N questions.
\begin{verbatim}
RAL GKS 1.38 ALPHA/VMS version
DO YOU WANT TO PLOT GAIN VS RADIUS? Y/N
\end{verbatim}

You will also be given the option of plotting the  gain vs. time.
on separate plots  and then of plotting
 
Figure 6 shows some of the graphs obtained by plotting x-ray laser gain versus distance for the 196 Angstrom 
line, while  Fig. 5 shows
the efective gains (spatially integrated) for some lines (including the 196 Angstrom) vs. time.
 
\newpage
\vspace*{9.15in}
Figure 1: FILP3 output - 
Log electron temperature vs. distance.
%set{counter}{}
\newpage
\vspace*{9.15in}
Figure 2: FILP3 output -
position of all cell centres plotted vs. time.
\newpage
\vspace*{9.15in}
Figure 3: ION3 output -
ion density vs. distance (for 4 ion stages)
\newpage
\vspace*{9.15in}
Figure 4: XRLREC3 output -
x-ray laser gain versus distance
\newpage
\vspace*{9.15in}
Figure 5: XRLREC3 output -
efective gains (spatially integrated) for Alpha, Beta and Gamma lines vs. time.
\newpage
\vspace*{9.15in}
Figure 6: XRLCOL3 output -
x-ray laser gain versus distance 
\newpage
\vspace*{9.15in}
Figure 7: XRLCOL3 output -
efective gains (spatially integrated) for some lines vs. time


\section{Summary}
It is hoped that this guide will provide enough detail for any
user to run MED103. If you have any problems contact The Theory Group.
The users are also advised to
keep an unmodified version of the code and to include all personal additions/modification in a new version.
When encoutering problems they should run the original version of the code and if the problem re-occurs,
then contact the Theory Group with a copy of the input data file.

If people are actively engaged in writing some piece of code that changes
the physics, 
this may be of interest to other users and we would like to hear about it. If some important
piece of physics is lacking
then of course, it would be our aim to include this in future versions of
MEDUSA. 
 
\begin{thebibliography}{9}
\bibitem{cpc}JP Christiansen, DETF Ashby and KV Roberts
{\sl MEDUSA A one-dimensional laser fusion code}
Computer Phys Comm {\bf 7} 271-287 (1974)
\bibitem{med101} PA Rodgers, AM Rogoyski and SJ Rose, RAL-89-127, Dec 1989.
\bibitem{djaoui1}A Djaoui and SJ Rose, J. Phys. B: At. Mol. Opt. Phys.
{\bf 25}, 2745-2762 (1992)
\bibitem{Bhatia}AK Bhatia U Feldman  and JF Seely 
{\it At. Data Nucl. Data
Tables} {\bf 32} 435, (1985)
\bibitem{djaoui2}A Djaoui and AA Offenberger, Phys. Rev. E, {\bf 50},
4961 (1994)
\bibitem{sjr}GJ Pert and SJ Rose, Appl. Phys. B, {\bf 50}, 307, (1990)
\bibitem{Luciani}J F Luciani {\it et} al, Phys. Fluids {\bf 28}, 835, (1985)
\end{thebibliography}
\newpage
\begin{center}
{\LARGE\bf APPENDICIES}
\end{center}
\appendix
\section{Arithmetic gridding}
A further modification made in MED103 is the option of arithmetic
gridding (this is where the thicknesses of the cells gradually
increases or decreases within a layer rather than remaining identical).
Each of the 3 layers may be altered independently. The
gridding is based upon the formula:
$$w_i = \left( {r_f^i -1 \over r_f^n -1} \right) \times S $$
$$ \Delta_i = w_i - w_{i-1}$$
$$ \Delta_1 = w_1$$
where $w_i$ represents the co-ordinate relative to zero of the
$i$th cell boundary,
$\Delta_i$ is the width of the $i$th cell,
$n$ is the number of cells in the layer, $i$ varies from 1 to
$n$ and $S$ is the thickness of the layer (i.\ e.\ RINI, DRGLAS
or DRPLAS).
The factor in brackets varies between 0 and 1 depending on $i$ and
the difference in cell thickness from one cell to the next is controlled
by $r_f$ (i.\ e.\ RF1, RF2 or RF3).
 
If $r_f$ is almost but not quite equal to 1.0 (e.\ g.\
0.99999) then the
gridding will be linear as in the old versions of MEDUSA.
 
If  $r_f$=1.0 the denominator is zero so the
code resets any input from 1.0 to 0.99999. This is also the
default so to obtain a linear mesh either $r_f$=1.0 or $r_f$=0.99999
can be used.
 
It is only possible to fix any three of the four quantities $r_f$,
$n$, $S$ and $\Delta_1$ (or any $\Delta_i$).
Suppose, for example, we require a layer containing 10
cells to have its first cell 10 times larger than its 10th cell. Using
the formula
$$r_f = \left( {\Delta_n \over \Delta_1 }\right)^{{1\over n-1}} $$
we find $r_f=0.7743$. Given $r_f$ and $n$ the thickness
of the layer may be specified. If we to specify the
thickness of the beginning and end cell
we lose control over the total thickness
of the layer. To calculate the total thickness of the layer (i.\ e.\ one
of RINI, DRGLAS and DRPLAS) use
$$ S = \left( {r_f^n -1 \over r_f -1} \right) \times w_1 $$
If, for example, we set $\Delta_1 = 10\mu$m and
$\Delta_{10}=1\mu$m then the total thickness of the layer
MUST be 40.87$\mu$m. Alternatively if we set
$S =  100\mu$m (with $n = 20$ and $r_f =0.8361$)
then $w_1=24.47\mu$m and $w_{10}=2.447\mu$m.
 
It is not possible to control the number of cells,
the ratio, the total thickness and
the individual thicknesses simultaneously.
 
This facility is very useful for a number of things. It can be used to
ensure that the zoning is at its finest at the edge of a target where
most detail is needed, without having to set up a large number
of different layers. It can be used to interface between layers
of differing cell width. This helps avoid jumps in the mass between
consecutive cells that may cause problems.
\newpage
\section{A fortran program for calculating RF1, RF2 and RF3}
\begin{verbatim} 
c  Program to calculate rf value when thickness of first zone,
c  total thickness of shell and number of zones are specified
      PROGRAM MAIN
      IMPLICIT DOUBLE PRECISION (a-h, m-z)
      CHARACTER answer*1
c  Inputting the data
      WRITE(6,1000)
1000  format(/1x,' Given the thickness of the first cell and the ' ,
     & 'total thickness, this code calculates the RF value',/1x)
 1    WRITE (6,*) 'Please enter Delta1(in any units): '
      READ (5,*) delta1
      WRITE (6,*) 'Please enter total thickness(same units as delta1):'
      READ (5,*) thickness
      WRITE (6,*) 'Please enter number of meshes:'
      READ (5,*) meshes
c  Is the data correct????
      WRITE (6,*) ' '
 3    WRITE (6,*) 'Is the above information correct (y/n)?'
      READ (5,FMT='(A)') answer
      IF (answer .EQ. 'y' .OR. answer .EQ. 'Y') GOTO 5
      IF (answer .EQ. 'n' .OR. answer .EQ. 'N') GOTO 1
      GOTO 3
c  Creating blank space
 5    DO i = 1, 1
         WRITE (6,*) ' '
      END DO
c  Doing the calculation
      rdm = 1.0e30
      DO 10 i = 10000, 990000
         rf = float(i)*1.e-05
         IF(rf .EQ. 1.000) GOTO 10
         rd = (ABS (delta1-thickness*(rf-1)/(rf**meshes -1))/delta1)
         IF (rd .LT. rdm) THEN
            rf2 = rf
            rdm = rd
         END IF
 10   CONTINUE
      rf = rf2
      rd = rdm
      del1=thickness*(rf-1.)/(rf**meshes-1.)
      deln=del1*rf**(meshes-1)
      delnm1 = del1*rf**(meshes-2)
      delnm2 = del1*rf**(meshes-3)
c  Displaying the results
      WRITE (6,*) '                          RESULTS'
      WRITE (6,*) '                          -------'
      WRITE (6,*) '          Delta1             = ', delta1
      WRITE (6,*) '          Total thickness    = ', thickness
      WRITE (6,*) '          Number of meshes   = ', meshes
      WRITE (6,*) ' '
      WRITE (6,4000)rf,del1,deln,deln/delnm1
4000  format( '  rf ',1pe14.7,'  Delta1 = ',e14.7,' Deltan = ',e14.7,
     &'  delta n/delat n-1   = ', e14.7)
      WRITE (6,5000)1/rf,deln,del1,delnm1/deln
5000  format( '  rf ',1pe14.7,'  Delta1 = ',e14.7,' Deltan = ',e14.7,
     &'  delta n/delat n-1   = ', e14.7)
      WRITE (6,*) ' '
      WRITE (6,2000)
2000  format(' Use rf .LT. 1 for meshes decreasing towards ',
     &'the laser beam (left to right)')
      WRITE (6,3000)
3000  format( ' Use rf .GT. 1 for meshes increasing towards ',
     &'the laser beam ')
      WRITE (6,*) ' '
c  Re-run the program???
      WRITE (6,*) ' '
 20   WRITE (6,*) 'Do you want to calculate another value (y/n)? '
      READ (5, FMT='(A)') answer
      WRITE (6,*) ' '
      IF (answer .EQ. 'y' .OR. answer .EQ. 'Y') THEN
         GOTO 1
      ELSEIF (answer .EQ. 'n' .OR. answer .EQ. 'N') THEN
         GOTO 30
      ELSE
         GOTO 20
      ENDIF
 30   STOP
      END
\end{verbatim}
\newpage
\section{Special features}



\end{document}
